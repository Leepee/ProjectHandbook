\chapter{Project Operation and Management}

Projects divide into the following broad catagories:

\begin{itemize}
    \item those that involve the \textbf{design and build} of a technical product
    \item those that involve experimental design based on techniques such as \textbf{computer-based simuilation or mathmatical modelling.}
    \item those that involve design and implementation of a \textbf{new or experimental technique.}
    \item those that involve the \textbf{testing and proving of a theory or concept} using scientific research techniques. 
\end{itemize}

As with other commercially-based and academic projects, the final year project consists of a number of key phases: 

\begin{itemize}
    \item itentification of the problem to be solved
    \item selection of suitable tools and methods with which to conduct the work
    \item conduct of the project work
    \item reporting on the process
\end{itemize}

Project management is about planning these phases over time, and ensuring that resources are available for each one.

Project monitoring and control are concerned with recording the progress of the project and using a pre-determined method to ensure that project milestones are completed on time.

While carrying out the four phases listed above, you will also be expected to demonstrate the ability to:

\begin{itemize}
    \item synthesise theory and practice in providing interesting and novel solutions
    \item justify the choices that are made at each stage of the project
    \item evaluate the fitness for purpose of the chosen solution
    \item evaluate whether the tools and methods used were appropriate
    \item reflect on each aspect of the development process
\end{itemize}

All of these skills are assessed at various times during the module.

There are two distinct areas within the project:

\begin{itemize}
    \item there is the “product” - this is what you actually create, and might be a hardware unit, a
    software system and associated documentation, a computer based simulation which
    solves a complex problem, or the results of a research investigation.
\end{itemize}

and

\begin{itemize}
    \item there is the “process” that you use to research, plan and monitor the creation of the
    “product”, including the selection of appropriate methods, tools and techniques.
\end{itemize}

As part of your final report, you will be expected to evaluate your “product” in terms of how well it satisfies your original objectives, and also to review the tools and techniques you used during the process.

In assessment terms, the process you use to conduct your project is at least as important as the product you create. It is therefore important that you maintain careful records in a logbook and library throughout the project to assist you in writing up the project process.

\textbf{One of the key elements for any report is style. You are strongly advised to write in the third person rather than the first person, using the passive voice. More information on this is provided on SOL and through the Library services on the Portal.}

\section*{Project mangement tools}

Undertaking good project management requires you to utilise project management tools and
demonstrate that you have followed the requirements of the organisation for which you are
working. The Media Technology Project has a minimum expectation of use and evidence of project
management tools which forms part of the assessment criteria for various stages of the project.
Evidence of project planning will form the appendices to the Definition report as follows:

\begin{enumerate}
    \item \textbf{Project Proposal}
    \subitem You are required to have an agreed project proposal, which is a 1-page summary of your project
    topic and objectives, which must be agreed and signed off by your supervisor. This forms part of
    the record of your project progress, and non-completion will have a negative effect on your grade
    for the project definition.

    \item \textbf{Project Supervision Record}
    \subitem It is a requirement to demonstrate sustained progress on your project and communication with your supervisor, and agree tasks and objectives. It is therefore a requirement that you complete a
    supervision record form each time you meet with your supervisor. This should be signed by both of
    you and kept by you – you must include these in the appendices for both definition and final report
    submission, with at least two completed for the Definition report. Non-completion will have a
    negative effect on your grade.

    \item \textbf{Time Plan}
    \subitem  All projects must demonstrably make use of time planning in the form of a Gantt chart, plus
    associated justification of the time plan. This is expected to be revised at different stages of the
    project to reflect changes in the project timing and updated objectives.

    A copy of the initial time plan should be included in the appendices to the Definition report, and
    copies of ALL time plans should be included with the Final report, with a discussion and justification
    of any changes in the time plan.

    \item \textbf{Ethical Review}
    \subitem All projects must be carried out within the framework of the University’s Ethics Policy, see section 2S of the Academic Handbook (available via the portal), and must address the following five questions.

    ALL projects must undergo an Ethics Review as part of the definition stage – projects without an Ethics Review will not be permitted to proceed. A copy of the Ethics Release form should be included in the appendices of the Definition Report and of the Final Report. Non-completion will have a negative impact on the grade.
    
    \item \textbf{Logbook}
    \subitem  It is good scientific practice to maintain a logbook throughout any project. This forms the basis for your write-up, is a place to document project progress in any form, and forms the basis of defence for any Intellectual Property cases you might be involved in, particularly if you are aiming to protect your IP. It is therefore a requirement for you to maintain a technical logbook, noting all progress made in the project. This should be brought to supervision meetings and signed by the supervisor at each meeting. Sight of this logbook forms part of the assessment criteria for the project management element of the Definition report. You will also submit the logbook with the final project report to demonstrate evidence of sustained project progress.
\end{enumerate}

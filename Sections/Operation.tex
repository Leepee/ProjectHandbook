\chapter{Project Operation and Management}

Projects divide into the following broad categories:

\begin{itemize}
    \item Those that involve the \textbf{design and build} of a technical product
    \item Those that involve experimental design based on techniques such as \textbf{computer-based simulation or mathematical modelling.}
    \item Those that involve design and implementation of a \textbf{new or experimental technique.}
    \item Those that involve the \textbf{testing and proving of a theory or concept} using scientific research techniques. 
\end{itemize}

As with other commercially-based and academic projects, the final year project consists of a number of key phases: 

\begin{itemize}
    \item Identification of the problem to be solved
    \item Selection of suitable tools and methods with which to conduct the work
    \item Conduct of the project work
    \item Reporting on the process
\end{itemize}

Project management is about planning these phases over time and ensuring that resources are available for each one.

Project monitoring and control are concerned with recording the progress of the project and using a predetermined method to ensure that project milestones are completed on time.

While carrying out the four phases listed above, you will also be expected to demonstrate the ability to:

\begin{itemize}
    \item Synthesize theory and practice in providing interesting and novel solutions
    \item Justify the choices that are made at each stage of the project
    \item Evaluate the fitness for purpose of the chosen solution
    \item Evaluate whether the tools and methods used were appropriate
    \item Reflect on each aspect of the development process
\end{itemize}

All of these skills are assessed at various times during the module.

There are two distinct areas within the project:

\begin{itemize}
    \item There is the “product” - this is what you actually create, and might be a hardware unit, a
    software system and associated documentation, a computer based simulation which
    solves a complex problem, or the results of a research investigation.
\end{itemize}

and

\begin{itemize}
    \item There is the “process” that you use to research, plan and monitor the creation of the
    “product”, including the selection of appropriate methods, tools and techniques.
\end{itemize}

As part of your final report, you will be expected to evaluate your “product” in terms of how well it satisfies your original objectives, and also to review the tools and techniques you used during the process.

In assessment terms, the process you use to conduct your project is \textbf{at least as important as the product you create.} It is therefore important that you maintain careful records in a logbook and library throughout the project to assist you in writing up the project process.

\textbf{One of the key elements for any report is style. You are strongly advised to write in the third person rather than the first person, using the passive voice. More information on this is provided on SOL and through the Library services on the Portal.}

\section*{Project management tools}

Undertaking good project management requires you to utilize project management tools and
demonstrate that you have followed the requirements of the organization for which you are
working. The Media Technology Project has a minimum expectation of use and evidence of project management tools which forms part of the assessment criteria for various stages of the project. Evidence of project planning will form the appendices to the Definition report as follows:

\begin{enumerate}
    \item \textbf{Project Proposal}
    \subitem You are required to have an agreed project proposal, which is a 1-page summary of your project topic and objectives, which must be agreed and signed off by your supervisor. This forms part of the record of your project progress, and non-completion will have a negative effect on your grade for the project definition.

    \item \textbf{Time Plan}
    \subitem  All projects must demonstrably make use of time planning in the form of a Gantt chart, plus associated justification of the time plan. This is expected to be revised at different stages of the project to reflect changes in the project timing and updated objectives.

    A copy of the initial time plan should be included in the appendices to the Definition report, and copies of ALL time plans should be included in the appendix of the Final report, with a discussion and justification of any changes in the time plan.

    \item \textbf{Ethical Review}
    \subitem 

    ALL projects must undergo an Ethics Review as part of the definition stage – projects without an Ethics Review will not be permitted to proceed. A copy of the Ethics Release form should be included in the appendix of the Definition Report and of the Final Report. Non-completion will have a negative impact on the grade, and can result in academic misconduct procedures against you. 

    \item \textbf{Risk assessment}
    \subitem This should be an assessment of \textit{project risk}. This can include risks to health and safety where they affect the viability or progress of the project (if there are health and safety risks in the project, they need to be analysed through a separate risk assessment). Use a proper template for the assessment, taking into account likelihood, impact, and any mitigation strategies that are applicable. 
    

\end{enumerate}

\chapter{Assessement}

The objective of the assessment of the project is to determine and appraise your ability to:

\begin{itemize}
    \item Diagnose and structure a problem
    \item Apply techniques and concepts from the course in a meaningful way
    \item Present logical results and conclusions deriving from analysis
    \item Express any limitations with the method used
    \item Demonstrate an understanding of problems involved woth the implementation of proposed solutions
    \item Communicate both in writing and orally, clearly and concisely
    \item Present professional and well formatted documentation
\end{itemize}

Project elements will be marked in accordance with the university assessment regulations. Project reports may be sent to external examiners, and the external examiners may visit to assess the general progress of the projects. 

\begin{tcolorbox}
    University regulations will apply for late submissions.
\end{tcolorbox}

\section{Assessment criteria}


As with all assessments, you will be given an outline of the criteria used by the assessors so that everyone has a clear idea about which aspects of the project and the report are considered to be the most important. These are shown in the assessment briefs which are available on SOL. Please read them carefully before starting work on your project.


\subsection{Second marking \& moderation procedure for project reports}

Due to the fact that the project report is responsible for such a large proportion of your degree classification, there is a more stringent marking procedure in place.

All project final reports are fully blind second marked (another assessor is assigned and the project report is marked as though it were a fresh unmarked report, and the other assessor does not know the grade).
\chapter{Project Selection}

The project process commences with the selection of a topic area, \textit{as early as possible}.

\begin{tcolorbox}
    Students will be encouraged to develop their own project topic, but several suitable topics have been identified by unit supervisors, for which you may 'pitch’. Those who have been unable to identify a suitable topic by the end of week three of the academic year may be allocated a title and supervisor.
\end{tcolorbox}
    
To make a start on the process of choosing a topic for the project, the following questions may be of help:

\begin{itemize}
    \item What aspects of the course would I like to pursue further in a practical way?
    \item How will my project topic selection relate to my degree pathway?
    \item Will my project satisfy the module outcomes?
    \item Can I choose a project that will help me to get my first / next job?
    \item What information and opportunities to talk to specific people and to enter other organisations would I like to have?
    \item What problem areas exist in my job, department or organisation that I would like to see tackled?
    \item Which industry-related problems do I consider as being important for investigation?
    \item What practical outcome would I like to see achieved as the result of a study and investigation?
\end{itemize}

Once the project has been identified, you will have to prepare a more detailed project \textit{definition}

\section{Intellectual Property Rights (IPR)}

The University Policy states that all work produced by students as part of their course remains the property of the student unless there is a written agreement to the contrary. Students enrolled with the University will be required to assign their IP to the University before they become involved in any activity in which the University may require use or control of the IP for teaching, research or commercialisation

The exception to the above is where it is clear that the IP rests with the 3rd party; such as a part-time student doing work for his or her employer.

\begin{tcolorbox}
    Your project is not a legally binding contract and under no circumstances does your project subject any of the Parties to liability for breach, whether material or minor, of contract or any other liability under national or international law or any other applicable law.
\end{tcolorbox}

If you do a project for a third party the expectation of the third party must be that you will not produce anything for them. If you do produce something that they can use that is to be considered a bonus.

The university will abide by reasonable non-disclosure agreements where company confidential or sensitive information is included in the project. The university \textbf{must} reserve the right to show all project documentation to the supervisor, second marker, project coordinator and external examiners and any member of the Academic Standards and Quality Service and Academic Misconduct panellists where such report is suspected of Academic Misconduct.

\section{Academic Misconduct}

Any submissions must be student’s own work and, where facts or ideas have been used from other sources, these sources must be appropriately referenced using Solent Harvard Referencing style. The University’s Academic Handbook includes the definitions of all practices that will be deemed to constitute academic misconduct. Students should check this before submitting their work.

Procedures relating to student academic misconduct are available from the student Portal.
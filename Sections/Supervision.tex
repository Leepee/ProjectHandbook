\chapter{Supervision}

Once the project proposal has been accepted it is your responsibility to carry out the project within the agreed time.

\begin{tcolorbox}
    The initiative to complete the project is expected to come from you, not your supervisor.
\end{tcolorbox}

Careful planning of the project is essential and after discussion with your supervisor you should produce an initial plan of methods, activities and timings in some appropriate form, a Gantt chart is the established recommendation.

Your initial plan may change as you get to grips with the investigation and shift the emphasis of your project, but such a plan will act as a project management tool, and it should be included in the project library. The Gantt chart should be included with each report and detailed methods and activities documented in your logbook.

\begin{tcolorbox}
    Your project supervisor will expect to see you on a regular basis (weekly, or once every 2 weeks) as conditions demand. Failure to attend these meetings will result in poor performance.
\end{tcolorbox}

The relationship between you and your project supervisor is founded in certain basic expectations placed on both parties. The role of the project supervisor is NOT, in any way, to carry out part of your project. Your project supervisor is there to help you make the most of the opportunity the project presents.

You will be expected to:

\begin{itemize}
    \item Provide regular updates on progress
    \item Communicate regularly
    \item Inform your supervisor of any problems that might arise that may affect your performance
    \item Discuss resource requirements such as hardware, software, laboratory access, etc.
    \item Listen to and act upon advice.
\end{itemize}

Your project supervisor will be expected to: 

\begin{itemize}
    \item Provide technical and academic advice as and when required
    \item Review progress and aim to ensure that you are setting and meeting appropriate objectives.
    \item Help you to develop the skills of research, investigation, and reporting commensurate to the expectations of a level 6 undergraduate student.
\end{itemize}

\begin{tcolorbox}
    It is not your supervisor’s job to check that you are carrying out work to the
    agreed timetable. You are expected to organise your own time and work to
    reach agreed deadlines as an integral part of this major work.
\end{tcolorbox}

You must allow plenty of time for delays in obtaining books and other information and in carrying out practical work. You must also remember to allow sufficient time for writing the final report - the time required for this is almost always underestimated!

Although the project is organised as a sequence of separate activities, there will be some overlap, and it may be necessary to return to earlier stages as your knowledge expands and you meet unexpected problems.

\section{Drafts}

The university regulations state the work cannot and should not be "pre-marked" - that is completed, read by a member of the academic staff and feedback given, and then handed in again for grading. This is an obvious concept designed to prevent work being completed and graded upon the merits of the lecturer rather than the student. 

For the project module however it is acknowledged that the scope and importance of the work merits an exception to this rule. Project reports, and to a lesser extent project posters, can be reviewed by the supervisor and corrections issued prior to handing in the work. You will not recieve a grade, or indicated grade as part of this process. The grading process takes into account more information than can be isolated by a single draft read, and takes more time. Consider this as a "polish pass" - intended to extract the best performance from the work.

There will be an opportunity on the SOL page for you to submit a draft, in which time you can reasonably expect a draft read and feedback. This remains, however, at the discretion and the time constraints of the supervisor. The upload for this draft will normally be issued around a month before submission, however if your supervisor specifically asks for it earlier, then that is at their discretion. 

\begin{tcolorbox}
    Draft reads are in excess of your usual expectations of lecturing staff and so are at the discretion of the supervisor and their time commitments. You can, however, reasonably expect a draft read during your project.
\end{tcolorbox}
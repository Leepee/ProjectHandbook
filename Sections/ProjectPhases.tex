\chapter{Project Phases}

The following summary tables should be read in conjunction with the notes in the remainder of this section.

The project management strategy is designed to reflect the practice of research and development in industry, and three assessment elements are used within the unit to reflect the core stages of project progress.

\begin{itemize}
    \item The initial phase of the project is used to define the problem to be solved, set out the scope and purpose of the project (including any business considerations where appropriate) and to assess potential solutions. This should result in the submission of a Project Definition in the form of a written report which defines the theoretical background to the project, sets project aims and objectives, identifies and evaluates key options for approaches and specifies a method which the student will adopt to solve the project ‘problem’.
    \item The second (main) phase of the project involves undertaking the work to solve the problem set out in phase one, using appropriate tools and techniques. At the end of the project a Project Report should be submitted. The project report should take the approach of technical documentation and objective analysis of the project from inception to completion. This includes describing and analysing the project methods used, reviewing and discussing the progress of the project against initial aims, objectives and planning, evidencing results of the project output from testing, assessing the fitness for purpose of project outputs and drawing overall conclusions regarding the success of the solution to the project problem.
    \item The final stage of the project is a project exhibition which will be used to provide the opportunity for students to disseminate the outcomes from their projects to a wider audience. This may involve discussions with other students and prospective employers. This will be an online exhibition, in which the student will present their project in the form of both an academic poster and a short video explaining key aspects of the project, evidencing the output and answering set questions on the project conduct. The student will be assessed on their ability to present and explain their project clearly and concisely.
    \item Subject to COVID-19 restrictions, Students will also present their project in the format of a ‘trade show’, with a stand exhibiting their project, supported by posters and other visual aids. This will be non-assessed, but an opportunity to present the project in a public forum with representatives from industry, members of the staff and public acting as customers at the exhibition.
\end{itemize}


\section{Brief summary of project phases}

\subsection{Phase 1 - Project Definition}

This happens in term one, during weeks 1 to 7, and you should budget a minimum of 85 hours of work time.

\subsubsection{Typical Activities}

The typical activities of this phase are:

\begin{itemize}
    \item Define the scope and context of the problem
    \item Identify and specify research questions
    \item Assess the market or customer needs and Develop a business case
    \item Define aims and objectives
    \item Specify requirements
    \item Study underpinning theory
    \item Propose possible solutions
    \item Identify risks and project unknowns
    \item Identify Ethical, professional and legal issues
    \item Derive initial plan/methodology
\end{itemize}

\subsubsection{Deadlines \& Deliverables}

By \textbf{29/10/2021} at the latest you should have produced and signed a Project Proposal (one-pager, not graded):

\begin{itemize}
    \item 1 page A4 - a brief summary of the project
    \item Include aims, objectives, proposed method.
    \item Signed by supervisor and student
\end{itemize}

The proposal should be a general description of the problem and proposed solution, with an outline of the aims and objective inherent. 

\medskip

\textbf{Failing to complete this in a timely manner means that you may be assigned a supervisor and/or a project.}

\medskip

By \textbf{7/10/2021} (assignment deadline) you should produce a Project Definition Report (20\%):

\begin{itemize}
    \item 2000 words
    \item background
    \item Options analysis
    \item Method statement
    \item Technical specification
    \item A copy of the Project Proposal
    \item Project supervision record
    \item Time plan
    \item Ethics review
\end{itemize}

The definition Report is used to build upon the idea of your project, and creates a proper plan for undertaking it. It should be a discussion of vthe project context and scope, deriving the aims and objectives by analysis of existing theory and background. You should examine existing systems or approaches to the problem, and provide an overview of how these approaches affect the project path, and which options are a good choice. You should produce a technical specification for the outcomes of the product or project, create a proposed method and project plan, and include supporting and subsidiary information such as ethical review, risk analysis. 

The report should be fully supported by good quality citations from the field of study.

\subsection{Phase 2 - Project Implementation \& Documentation}

This happens throught the year, after the Definition submission, and is the phase concerning your actual project work. It is recommended that you spend a \textbf{minimum} of 200 hours, if your project does not comprise of this level of commitment, it likely needs to be revised.

\subsubsection{Typical Activities}

The typical activities of this phase are: 

\begin{itemize}
    \item Specify system requirements and high level design or
    \item Select appropriate tools and methods to be applied
    \item Design and implement according to project process model
    \item Fully document design and build process
    \item Conduct tests
    \item Analyse performance against initial specification
    \item Write up project report.
\end{itemize}

\subsubsection{Deadlines \& Deliverables}

By \textbf{6/5/2022} (assignment deadline) you should submit your Final Report (60\%):

\begin{itemize}
    \item 10,000 words ($\pm 10\%$)
    \item Background, survey of literature and the state-of-the-art
    \item Full detail of project development, implementation, testing, and results
    \item 2 bound copies to be submitted as well as an online copy
    \item Submit project title and synopsis for project exhibition
\end{itemize}

The project report itself is documentation of the project process, from the inception of the idea, through background reasearch, detailed method, and results an analysis. It should also include an evaluation section detailing the limitations that were found during the researc, and a discussion and conclusion giving analysis of the results and their implications. 

\subsection{Phase 3 - Project presentation}
This happens toward the end of the academic year, and is typically the last assessment you'll do. It's recommended to spend a minimum of 15 hourso nthis part of the project:

\begin{itemize}
    \item Design an academic stlyed poster for presentation
    \item Make any final adjustments to product
    \item Adapt or create any live demos that might be appropriate
\end{itemize}

\subsubsection{Deadlines \& Deliverables}

By \textbf{20/5/22} (assignment deadline) you should produce an A1 poster to present on Poster day (exhibition date to be confirmed).

\section{Phase one: Project definition}

The Project Definition phase consists of the production of a project proposal and a project definition report, which will include a project specification and a detailed time plan.

\subsection{Starting the project: The Project Proposal}

For your Project Proposal, you will need to propose a project topic and provide an
overview of the project explaining the background to the idea and some initial aims and objectives. This will take the form of a single page outline of your project.

The proposal is designed to ensure that if the project is completed according to plan, it will meet the criteria for level 6 of an undergraduate course. You must find an appropriate supervisor for the project and they must sign off on the initial proposal.

The final choice of project will be the result of a four-way consultation process between you, the project co-ordinator, your project supervisor and (where  appropriate) your employer or other industrial partner.

The outline proposal is not marked, but forms the basis of your project agreement with your supervisor. You must not start your project until the topic has been agreed and you have a supervisor.
You should try to start the process off as soon as possible, and the outline proposal should be signed off by your supervisor and handed to the project co-ordinator by the end of the first three weeks of the unit.

\begin{tcolorbox}
    Students who have not agreed and signed proposals by the 29\textsuperscript{th} of October may be assigned a project idea and supervisor. 
\end{tcolorbox}

The outline proposal should consist of no more than one side of A4, with the following sections:

\begin{itemize}
    \item Student name
    \item Project title
    \item Description of proposed project method
    \item Aims and outline objectives
\end{itemize}

You should include a space at the bottom for both you and the supervisor to sign. 

\begin{tcolorbox}
    You have not secured a supervisor for your project until both you and the supervisor have agreed the topic, and the proposal is signed and submitted. Supervisors will not reserve spaces for individuals without signed proposals, and will operate on a first-come, first-served basis.
\end{tcolorbox}


\section{Assessment Element one: The Definition Report}

The Project Definition document should explain, in detail, what the project is expected to achieve and identify the options (or possible solutions) that may lead to the desired outcomes. Essentially it can be expressed as the  question; “What am I going to do?”.

As this phase of the project is also about feasibility, it addresses some issues of enabling technology and systems. In other words it answers the questions; “Are there methods, technologies and systems that can be used to solve this problem?”, and if so, “What is my strategy for using them?”.

Your project definition requires background research to determine the nature of the problem, and to investigate potential solutions. This research must include an analysis of the theoretical background for the project and an investigation to establish what work has already been carried out in the chosen subject area.


The result of your background research should provide the reader with a summary of the technical background for your project, so that they can understand the decisions that you make in the options analysis and conclusion/method statement. It should also give a ‘paper trail’ of where you have sourced your information from, by appropriately citing all sources of information, demonstrating that your work is based on good, current research. 

All parts of your reports should cite references, not just the background. Where information sources
are not cited, this may result in a charge of plagiarism, and will certainly affect your grade, as research
and theory is core to all grading criteria.

Doing the research for the background also enables the setting and justification of clear aims and objectives for the complete project.

The report should demonstrate that your project is viable and should give both you and your supervisor confidence in your ability to complete the project successfully. The report should be no longer than 2,000 words/10 pagesincluding references. It should demonstrate that you have already done a considerable amount of work (minimum 100 hours).

You should evidence use of project management tools in your report, including copies of your signed project proposal, risk assessment, ethics release, and time plan (Gantt chart). These form evidence of project management, and therefore are important to the assessment criteria. 

The report follows the normal structure of most technical reports and indicatively \textit{could} contain the following sections:

\begin{enumerate}
    \item Introduction
    \item Aims and objectives
    \item Background
    \item Options analysis
    \item Conclusion and method statement
    \item References
    \item \textbf{Technical specification}
    \item \textbf{Risk assessment}
    \item \textbf{Ethics statement}
    \item \textbf{Time plan}
    \item \textbf{Project proposal}
\end{enumerate}

Items in \textbf{bold} can be placed in the appendices. Please do not place anthing else in the appendices. 

\begin{tcolorbox}
    The definition is an important assignment in the Media Technology Project module, and is worth 20\% of the final grade. It bust be submitterd by DATE in PDF format \textit{only}. 
\end{tcolorbox}

\section{Assessment Element two: Implementation, Documentation, and Evalution}

Before you start doing any work on the “product”, you should ask yourself the following questions:

\begin{itemize}
    \item When I get to the end of the work, how am I going to test whether what I have produced matches my aims and objectives?
    \item How am I going to just whether the method I used to complete the project was the best one? 
\end{itemize}

This process is likely to involve practical work such as conducting technology proving or a pilot survey. This can be something fast and dirty, and it doesn't have to triel the entire project idea, perhaps just the more contentious or unreliable elements. 

This phase of the project is about applying the tools, methods and techniques previously selected, to the solution of the problem, in an effective way. This may involve design, build and test, systems modelling or experimental work in a business context.

The final part of the project is about evaluating both the “product” and the “process”, and can be expressed as a series of questions:

For the "product"

\begin{itemize}
    \item Is this result reliable and fit for purpose?
    \item Does what I produced match with my original intentions and expectations?
    \item How can I demonstrate the extent to which it does? 
    \item Does it fulfil the technical specification?
\end{itemize}

For the "process"

\begin{itemize}
    \item Could this be doen in a better way?
    \item Where did I have to move away from what I originally planned to do?
    \item Why did I have to? 
    \item Could I have forseen the problems?
    \item Should I have?
    \item How did I deal with these problems? 
    \item What was the impact to the integrity of the project?
\end{itemize}

Answering these questions involveds comparing the original theory, which led to the selection of your approach, to the experiance and knowledge gained through the conduct of the project

Your final report is the complete writeup of the project from start to finish. It must be a stand-alone
report – do not expect that a reader will be aware of your project or have read your definition. It
must therefore introduce the project concept and aims before going into the detail of how the project
was implemented.

\begin{tcolorbox}
    Two bound copies of the project final report must be submitted to the assignment office. An electronic copy must also be submitted by 4pm on the same date.
\end{tcolorbox}

The final report of around 10,000 words ($\pm$10\%) should be prepared in accordance with the published tecnical document guidelines in the appendix. The main body of the text will likely contain the following: 

\begin{enumerate}
    \item \textbf{Abstract}
    \subitem This should be a short synopsis ($\approx$250 words) of the project aims and what was achieved.

    \item \textbf{Introduction, Aims and Objectives}
    \subitem This should introduce the project to someone who is finding out about this project for the first time, and should re-iterate and clarify the aims and objectives of the project.

    \item \textbf{Background}
    \subitem This should provide a detailed overview of underlying theory and practice which relates to the project topic, for example relating to methodologies and enabling technologies. It should cite a wide range of appropriate literature sources to support the arguments made in the report. This chapter should enable a technically literate reader who is not familiar with your project topic to understand the decisions which you make in the methodology section of the report.

    \item \textbf{Method}
    \subitem This should analyse the methods used to undertake the project. You should also evidence and discuss the design of the project. The nature of the design will depend upon the type of project being undertaken. You should include supporting evidence for the design phase, which could include diagrams or tables.

    \begin{tcolorbox}
        You should avoid the use of 3rd party diagrams wherever possible. Redraw them and acknowledge their use with a "Derived from:" citation. Avoid the use of poorly taken photographs where diagrams would tell a better story.
    \end{tcolorbox}

    \item \textbf{Implementation and Results}
    \subitem You will need to provide evidence of the implementation of your project and discuss the issues and problems that arose during the implementation stage. You will need to discuss the outcomes of implementation whether for a physical system, theoretical concept or experimental work.

    The nature of testing of your project will depend again on the nature of the project. Forexample you would discuss the tests made on a design and build project, whether physically implemented or conceptual, to establish whether the system has met its specification. For a more experimentally based project you would discuss the results obtained from experimentation.

    \textit{Please} use screenshots rather than camera images where you need to show software. Present results using graphs and tables, formatted using an appropriate software package (e.g. Excel, SPSS, GnuPlot, or any number of proper plotting languages and packages).

    \item \textbf{Evaluation and Conclusions}
    \subitem You must evaluate the outcomes of your project. To do this you must analyse your system or
    experiments, tests and results and use your previously developed criteria for measuring whether the project has met its original objectives. (It is not as simple as saying that you enjoyed the project and it worked!)
    
    \item \textbf{Recommendations}
    \subitem Based upon your evaluation of the system and project as a whole you should make recommendations about the project. For example how would you have done it differently if you were to do the project again? What improvements could be made to your system or experiment if you were able to continue further work on your project? etc.

    \item \textbf{Ethics, conflict of interest, and/or collaboration statement}
    \subitem It is good practice to include a statment acknowledging the requirements of ethics, and highlighting any ethical considerations. This can be very short! Also a statement on conflicts of interest if working with external partners, or if you are investigating something important to a funder.

    \begin{tcolorbox}
        Where a project is undertaken in collaboration with an industrial academic partner or develops previous work, your report must include an explicit statement of your individual contribution to the work and the extent of the collaboration.
    \end{tcolorbox}
    
    \item \textbf{References}
    \subitem You must include a reference list at the end of your report, before the appendices. This must be in the correct Harvard referencing format. You must make use of these references to support your discussion in all parts of the report. (If you just include a reference list at the end you will likely fail your project). However don’t forget that all words and work must be your own and not direct quotes from third party sources.
    
    \item \textbf{Appendices}
    \subitem Appendices should NOT contain any text relating to the conduct of the project. They are only
    to be used for information which the reader will find useful but not critical to understanding the project. You should not put all of the images from the project into the appendices (this seems to be a holdover from where you might print the appendix only in colour to save money)
\end{enumerate}


\subsection{General notes}

The objective of the report is to indicate clearly the worth of the investigation by discussing the
research and investigative methods and the analysis and synthesis embraced by the "design and implement" aspects.

The assessors will be looking for clear links between the results analysis, conclusions, recommendations and implementation plans, with each one building on the previous stage of the argument. If the structure and content of the report fail to show these substantiating links, then no matter how brilliant, innovative or acceptable the recommendations, the project submission will not have achieved its overall objective.

The aim of the project report is to convince the assessors that:

\begin{itemize}
    \item the stated objectives of your project are in context and make sense
    \item you are aware of the extent to which the methodology used for the implementation/investigation was appropriate
    \item you are aware of any inherent flaws and the effects of divergences from your plan
    \item you know and have shown the strength or weakness of the evidence on which you base your recommendations
    \item well-chosen sources have been used and sufficient depth of reading has been
    undertaken, and identifiably used, in your investigation. (Sources must be referenced
    within the text).
\end{itemize}

You must discuss the structure of your report, in detail, with your supervisor. In general, reports should follow the structure described above. Adaptations to this structure may be necessary but should be kept to a minimum, and should have the agreement of your supervisor.


\subsection{Report length}

The report should be around 10,000 words in length. Being able to express information clearly and succinctly is an important business skill. Reports which are more than 10\% over the word limit will be penalised on the reporting category of the assessment criteria.

\section{Assessment Element three: Project exhibition}

The Project Exhibition will also form part of the assessment and provides an opportunity for you to display your work and demonstrate that you can clearly explain the project verbally. This will be scheduled after submission of the Final Report, in the last week of the academic year.

As part of the assessment for the project, the student must present the results of their project visually and verbally. The student will need to explain the project concept, approach and results in a visual and engaging format using an academic poster. This provides an opportunity for the student to succinctly explain the project, and their key results and also articulate aspects of the project process verbally.

Your poster should be formatted in A1 size, designed to be a summary of your project and must include:

\begin{itemize}
    \item The project title
    \item Your name and supervisor's name
    \item Outline information, formatted in similar sectioning to the main report
    \item Results and conclusions.
\end{itemize}

Posters should be readable from a 2 m distance, and contain roughly 400 words. Try to clearly present data, and include a system diagram where appropriate. Posters are largely subjective, but should bne presented in line with typical academic style.

You may also prepare a demonstration of your project for use on the day of the project, but you must be aware of power, noise, and light limitso n the environment. All project electronics must also be PAT tested by the university prior to the event.
\chapter{Document Preparation}

You must discuss the structure of your report, in detail, with your supervisor. In general reports should follow the structure described below. Modifications to this structure may be necessary but not to a great extent, and should have your supervisor’s agreement.

\section{Project Report Structure Overview}

The project is to be submitted on SOL in a machine readable PDF format (not a rasterised image of text). The report should take the following structure:

\begin{itemize}
    \item Front card cover (obtainable when binding the project)
    \item Cover page (Appendix C)
    \item Title page (Appendix D)
    \item Front matter
    \subitem Acknowledgements
    \subitem Abstract
    \subitem Table of contents
    \subitem List of figures
    \subitem List of Tables
    \subitem Glossary / acronyms
    \item Main body of text
    \item Backmatter
    \subitem References
    \subitem Bibliography
    \subitem Appendices  
\end{itemize}

Reports must be submitted in accordance with the University regulations.

\section{Project Structure}

The following is the recommended structure of your report in more detail.

\subsection{Title page}

Following the exemplar given in Appendix D, the title page should contain the following, in the following order:

\begin{enumerate}
    \item The name of the University
    \item The faculty of specialisation
    \item The award for which the project is submitted
    \item The academic year of original submission
    \item The name of the author
    \item The title
    \item The name of your supervisor
    \item The date of presentation
\end{enumerate}

You must include a statement at the foot of the title page that reads "This report is submitted in partial fulfilment of the requirements of Solent University for the degree of [full title of degree concerned]."


\subsection{Abstract}

This should not be more than half a page in length, and should be designed to provide the reader with a clear and rapid understanding of the aim of the project, as well as the key results and findings from the work.

\subsection{Table of contents}

This should be entitled 'CONTENTS' and is a statement of the main headings under which the text of the report is arranged. Page numbers should be those on which sections start and it is suggested that inclusive numbering, e.g. '22-23' is not used.

In the 'CONTENTS' the section titles listed should appear in the same type case as used to designate them in the text, i.e. Upper Case for main headings, Lower Case for sub-headings. 

Where there are more than five illustrations, a list of figures should be included. This will be entitled 'FIGURES'. Figure titles in this list should not be in capitals and the word 'Figure' should not be repeated each time. The same rules apply for a list of tables. 

The 'CONTENTS', 'FIGURES', and 'TABLES' should begin on an unnumbered page. Each element should be given a full page.

\subsection{Acronyms or Glossary}

All acronyms or technical terms need to be defined in an ordered list as well as being defined at the first point of use in the text, again on an un-numbered page.

\subsection{Main body text}

The main body text should be the first Arabic numbered section, with all prior parts using roman numerals. Arabic numbering should commence with the introduction section. It is suggested that page numbers should be consecutive and appear at the bottom centre of each sheet. Do not write ‘page’ before each number.

\subsubsection{Uses of secondary sources}

A major requirement of the project is that you justify decisions you make by referring to current thinking and practice. To avoid the charge of plagiarism, it is essential that any ideas, arguments, evidence, quotations, illustrations etc. that you take from the work of others are fully acknowledged by the use of references. This includes where you have paraphrased information (written it in your own words). You should still acknowledge where you got the original information from, and how you have ensured that it is correct and current.

\begin{tcolorbox}
    Under the University’s Academic Misconduct regulations, failure to make this acknowledgement is defined as \textit{plagiarism}. Penalties are applied to students who plagiarise the work of others, and these penalties are defined in the current version of the university’s “Academic Misconduct Procedures”. This document can be found on the student portal, and a link to it can be found in the assignment brief.
\end{tcolorbox}

All references should be cited using the Harvard Referencing System. If you are not sure how to use it, a number of resources are available from the library, the Portal and the SOL module.

\subsection{Appendices}

Appendices are used for items which could benefit a reader’s understanding of the project, but are not critical to it, or which if placed in the main body of the text would interrupt the flow of the report.

Examples include project management plans, supervision records, Ethics release forms, commented computer code, complex datasets, full system schematics/circuit diagrams.

Appendices should be lettered A, B, C etc.


\section{Proof Reading}

Your assessors will expect you to proofread your own report carefully before handing it in. This is easily the most important piece of work you complete at University, and it should reflect a professional attitude. Remember that the ability to communicate clearly and concisely in writing is one of the main Learning Outcomes of this module, and so errors \textit{do} make a grade difference. 

It is reasonable to ask other people to help you proof read, but they should not have an academic input into your writing. The same is true for commercial entities offering it as a service. If it is found that the report has been written or influenced by an external party this would be an act of plagiarism.

\section{Document Style}

The document should follow some style guidelines. It is important that it has your own overall style, but that it follows some rules to make sure it is or the right theme, and has the correct academic tone. This document is designed to serve as an example of the style, but the technical details are different as it's a different kind of document (handbook vs. academic paper). 

Check the style with your supervisor as they will have preferences that they will need to approve, but the following guidelines apply in principle:

\begin{itemize}
    \item 1 column of text
    \item Margins should allow around 60-70 characters of text per line
    \item Text should usually be left aligned, especially if using MS Word. If using LaTex, it will mostly do a better job of justifying text blocks.
    \item Choose a typeface. The normal recommendation is Trebuchet/Calibri but these fonts show a lack of engagement with this part of document preparation. Choose a typeface that is:
    \subitem Clean \& professional
    \subitem Usually serif
    \subitem Favoured to readability
    \subitem Without overtly flamboyant swashes
    \subitem Agreed upon with your supervisor!
    \subitem You can't go too far wrong with \hyperref[https://github.com/alerque/libertinus]{Linux Libertine} or \hyperref[https://tug.org/FontCatalogue/biolinum/]{Biolinum}, \hyperref[https://tug.org/FontCatalogue/computermodern/]{Computer Modern}, \hyperref[https://freefontsfamily.com/helvetica-font-family/]{Helvetica}, or \hyperref[https://fonts.google.com/specimen/Roboto]{Roboto}. Trebuchet or Calibri are actually fine too.
    \item Titles can be sans-serif even with serif main body typefaces
    \item Titles should be numbered, with main titles e.g. "1 Introduction", subtitles e.g. "1.1 Aims and Objectives" and subsubtitles e.g. "1.1.1 Aim"
    \item Titles and subtitles should be title case, lower titles in sentence case. 

\end{itemize}

Your supervisor will be the one reading the paper, so discuss this with them to make sure they are happy.
